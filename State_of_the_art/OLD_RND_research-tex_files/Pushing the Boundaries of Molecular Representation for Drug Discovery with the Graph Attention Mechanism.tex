\documentclass[12pt]{osa-supplemental-document}
\setboolean{shortarticle}{false}

\title{Pushing the Boundaries of Molecular Representation for Drug Discovery with the Graph Attention Mechanism}
\author{Xiong, Zhaoping & Wang, Dingyan & Liu, Xiaohong & Feisheng, Zhong & Wan, Xiaozhe & Li, Xutong & Li, Zhaojun & Luo, Xiaomin & Chen, Kaixian & Jiang, H. & Zheng, Mingyue. (2019). Pushing the boundaries of molecular representation for drug discovery with graph attention mechanism. Journal of Medicinal Chemistry. 63. 10.1021/acs.jmedchem.9b00959} 

\begin{abstract}
Hunting for chemicals with favorable pharmacological, toxicological, and pharmacokinetic properties
remains a formidable challenge for drug discovery. Deep
learning provides us with powerful tools to build predictive
models that are appropriate for the rising amounts of data, but
the gap between what these neural networks learn and what
human beings can comprehend is growing. Moreover, this gap
may induce distrust and restrict deep learning applications in
practice. Here, we introduce a new graph neural network
architecture called Attentive FP for molecular representation that uses a graph attention mechanism to learn from relevant drug
discovery data sets. We demonstrate that Attentive FP achieves state-of-the-art predictive performances on a variety of data sets
and that what it learns is interpretable. The feature visualization for Attentive FP suggests that it automatically learns nonlocal
intramolecular interactions from specified tasks, which can help us gain chemical insights directly from data beyond human
perception.
\end{abstract}

\setboolean{displaycopyright}{false} %copyright statement should not display in the  supplementary document

\begin{document}

\maketitle

\section*{Keywords}

deep learning; molecular representation; graph neural networks


\section*{Summary}
The goal of this paper is offer Attentive FP for molecular representation, it is a new graph neural network architecture which uses graph attention mechanism.

For molecular structures, efficient medicinal chemistry is dependent on associative thinking and pattern identification.
ecause molecular structures are typically made up of many-body interactions and complex electrical configurations, developing a full depiction of them is a difficult task. 

\subsection*{Past methods}
A vast amount of data on the biological impacts of chemical substances has been gathered and made publicly available in recent years.
\begin{enumerate}
    \item  Thanks to recent rapid advancements in high-performance computing, high class quantum chemical calculations have benefited many researchers.
    \item Over 5000 molecular descriptors have been created up till now.
    \item Feature engineering for these chemical descriptors has been the focus of traditional machine learning algorithms for QSAR/QSPR. Their purpose is to pick a subset of relevant descriptors to use in model building. There are two types of molecular representations: graph-based and geometry-based.
    \item In graph-based approach only information on the topological arrangement of atoms is used.
    \item Molecular geometry information, such as bond lengths, bond angles, and torsional angles, is used in geometry-based representations.
\end{enumerate}

\section*{Solution}
\begin{enumerate}
  \item In this work graph attention mechanism is used. A method can use an attention mechanism to focus on task-relevant elements of a neural network.
  \item The basic idea behind employing the attention mechanism on a graph is to get a context vector for the target node by focusing on its neighbors and immediate surroundings. 
  \item The model is called Attentive FP.
\end{enumerate}
\section*{References} 
 REFERENCES
(1) Schneider, G. Mind and Machine in Drug Design. Nat. Mach.
Intell. 2019, 1 (3), 128−130.

(2) Gaulton, A.; Hersey, A.; Nowotka, M.; Bento, A. P.; Chambers,
J.; Mendez, D.; Mutowo, P.; Atkinson, F.; Bellis, L. J.; Cibrian-Uhalte, ́
E.; Davies, M. The ChEMBL Database in 2017. Nucleic Acids Res.
2017, 45 (D1), D945−D954.
(3) Huang, R.; Xia, M.; Sakamuru, S.; Zhao, J.; Shahane, S. A.;
Attene-Ramos, M.; Zhao, T.; Austin, C. P.; Simeonov, A. Modelling
the Tox21 10 K Chemical Profiles for in Vivo Toxicity Prediction and
Mechanism Characterization. Nat. Commun. 2016, 7, 10425.
(4) Kuhn, M.; Letunic, I.; Jensen, L. J.; Bork, P. The SIDER
Database of Drugs and Side Effects. Nucleic Acids Res. 2016, 44 (D1),
D1075−D1079.
(5) Bolton, E. E.; Wang, Y.; Thiessen, P. A.; Bryant, S. H. PubChem:
Integrated Platform of Small Molecules and Biological Activities.
Annu. Rep. Comput. Chem. 2008, 4, 217.
(6) Wang, Y.; Xiao, J.; Suzek, T. O.; Zhang, J.; Wang, J.; Zhou, Z.;
Han, L.; Karapetyan, K.; Dracheva, S.; Shoemaker, B. A.; Bolton, E.
PubChem’s BioAssay Database. Nucleic Acids Res. 2012, 40 (D1),
D400−D412.
(7) Butler, K. T.; Davies, D. W.; Cartwright, H.; Isayev, O.; Walsh,
A. Machine Learning for Molecular and Materials Science. Nature
2018, 559 (7715), 547−555.
(8) Altae-Tran, H.; Ramsundar, B.; Pappu, A. S.; Pande, V. Low
Data Drug Discovery with One-Shot Learning. ACS Cent. Sci. 2017, 3
(4), 283−293.
(9) Gupta, A.; Zou, J. Feedback GAN for DNA Optimizes Protein
Functions. Nat. Mach. Intell. 2019, 1 (2), 105−111.

\end{document}